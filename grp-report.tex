\documentclass{nus-student-qe}

\usepackage{lipsum}

\author{John Smith}
\supervisor{Zachary Thomas}

\title{Lion Without A Home}

\department{School of Computing}
\publishdate{9}{August}{2017}

\begin{document}
	\maketitle
	\afterpage{\blankpage}
	
	\abstract
	\lipsum[1]
	
	\acknowledgments
	\lipsum[1]
	
	\tableofcontents
	\pagebreak
	
	\figures
	\tables
	\algorithms
	\listings
	\pagebreak
	
	\chapter{Introduction}
	\lipsum[1-4]
	
	\chapter{Second chapter}
	\lipsum[1]
	
	\section{First section}
	\lipsum[1]
	
	Image below (Figure \ref{fig:nus-logo}) is quite cool \cite{bellman}.
	
	\begin{figure}[!ht]
		\caption{NUS logo}
		\label{fig:nus-logo}
		\centering
		\includegraphics[scale=0.5]{images/nus-logo}
	\end{figure}
	
	\subsection{First subsection}
	\lipsum[1]
	
	\begin{table}[!h]
		\caption{Multiplication table}
		\label{table:mul-table}
		\centering
		\begin{tabular}{|*{18}{c|}}\hline
			-- & 1 & 2 & 3 & 4 & 5 & 6 & 7 & 8 & 9 & 10 & 11 & 12 & 13 & 14 & 15 & 16 & 17 \\\hline
			1  & 1 & 2 & 3 & 4 & 5 & 6 & 7 & 8 & 9 & 10 & 11 & 12 & 13 & 14 & 15 & 16 & 17 \\\hline
			2  & 2 & 4 & 6 & 8 & 10 & 12 & 14 & 16 & 18 & 20 & 22 & 24 & 26 & 28 & 30 & 32 & 34 \\\hline
			3  & 3 & 6 & 9 & 12 & 15 & 18 & 21 & 24 & 27 & 30 & 33 & 36 & 39 & 42 & 45 & 48 & 51 \\\hline
			4  & 4 & 8 & 12 & 16 & 20 & 24 & 28 & 32 & 36 & 40 & 44 & 48 & 52 & 56 & 60 & 64 & 68 \\\hline
		\end{tabular}
	\end{table}
	
	\begin{algorithm}[!h]
		\caption{Pseudocode example}
		\label{alg:pseudocode}
		\begin{algorithmic}
			\Function{IsPrime}{$N$}
			\For{$t \gets [2; \lfloor\sqrt{N}\rfloor]$}
			\If{$N \bmod t = 0$}
			\State\Return \textsc{false}
			\EndIf
			\EndFor
			\State\Return \textsc{true}
			\EndFunction
		\end{algorithmic}
	\end{algorithm}
	
	\chapter{Conclusion}
	\lipsum[1]
	
	\printmainbibliography{bibliography}
	
\end{document}
